\documentclass[english]{article}
\usepackage[T1]{fontenc}
\usepackage[latin9]{inputenc}
\usepackage{graphicx}
\usepackage{babel}
\begin{document}

\title{AXA Data Challenge}


\author{Aexandre Carton, Cl�ment Fischer, Alexandre Guinaudeau}

\maketitle

\section{Introduction}

In this report, we describe how we constructed a model to predict
the number of incoming call to the AXA french center. The dataset
on which the prediction is made consists in training data from the
years 2011 and 2012, giving the number of calls depending on the date
and a number of parameters such as the area of expertise of the call
and the state of the call center. It comes up with another complementary
set of data concerning weather information in France, which is supposed
to have an influence over the number of accidents and thus incoming
calls to AXA. We divided our work in 4 parts. First, the dataset has
to be preprocessed to be used efficiently. Then we use this data to
create features for our model. After this feature engineering step
we train a model and use it to predict the number of calls asked in
the submission file. We also evaluate our model using cross-validation
techniques.


\section{Preprocessing}

The data comes in two types of csv files : meteo\_2011.csv, meteo\_2012.csv
and train\_2011\_2012.csv. The main training dataset contains multiple
informations on the incoming calls to AXA's centers in France. The
number of incoming calls we have to predict, CSPL\_RECEIVED\_CALLS
is one of the 86 columns of the file. For each value of DATE and ASS\_ASSIGNMENT
(the field of competence to which the call is assigned), the model
has to predict the number of incoming calls for the three next days.

We used pandas library to read the files as databases objects. The
main advantage of pandas' read\_csv function is that it allows us
to read only the columns we are interessted in, thus saving much computation
time. 

For the training dataset, we keep the columns 'DATE', 'ASS\_ASSIGNMENT',
'CSPLRECEIVED\_CALLS' and 'DAY\_OFF'. We use this last column to eliminate
data corresponding to non worked days in AXA. We then group the data
by summing the number of calls having the same date and assignment
values. 

The meteo dataset consists in rows giving information at a given date
for a some location - city and departement ($\simeq$region) number
- in France. From this, we extracted the number of french departements
undergoing negative temperaures and number of them with rain, the
mean of tje lowest temperature over all departements.


\section{Feature engineering}

The data is then gathered in a matrix. The preprocessed csv files
are loaded into a pandas dataframe. We optimized the creation of features
by creating a class FeatureFactory with a general wrapper function
that works on a column of the feature matrix. To avoid bugs linked
with missing values in the preprocessed dataset, we added zeros to
the original training set, assuming there is no call when no data
is registered. The meteo missing values are also replaced with the
mean of the data.

\begin{figure}
\begin{centering}
\includegraphics[scale=0.5]{\string"C:/Users/HP/Desktop/AXA challenge/plot\string".jpg}
\par\end{centering}

\caption{Number of calls depending on the hour of the day.}


\end{figure}


To first have an idea of the features we selected, we plotted the
variation of the number of calls depending on what we expected to
have an influence. The features that we create are mostly dealing
with the date, ie. hour, day in the week, weekend or not, month, year.
Indeed, time appears to have essential of the influence over the number
of calls. The assignment is also a feature. Finally we added a few
meteo features, those we choose to extract at the preprocessng step.


\section{Training models}

We used the sklearn library to train a model over the features we
extracted. Our approach was the following : we try one of the regression
models among those provided in the library. From this we compute a
cross validation score one one full day. We repeat this until we find
out whiwh is the best model to be used. 

The best results we got with this aproach are a score of 113 with
the submission file. We then realised that this score drops to 75
by simply uploading a mean of all number of call values on the same
day of the week/hour/assignment and not using the regression algorithms.
Thus, our final algorithm connsists in : 
\begin{itemize}
\item Computing the means
\item Substract them from the data and train a model using this dataset
without means
\item Finally, add the predictions from the model to the means computed
first\end{itemize}

\end{document}
